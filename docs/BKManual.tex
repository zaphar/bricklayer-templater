\documentclass[letterpaper,10pt,oneside]{book}
\makeindex
\usepackage[latin1]{inputenc}
\usepackage[american]{babel}
\usepackage[T1]{fontenc}
\usepackage[dvips]{graphicx}

\usepackage[dvips]{hyperref}
\title{BrickLayer Development Manual}
\author{Jeremy Wall (Zaphar)}
\date{2005-12-05}

\begin{document}
\tableofcontents
\chapter{Overview}
\section{Intro}
BrickLayer is a set of tools to make designing applications for the web easy. Bricklayer is not a complete application. Neither is it an API. It is a Framework for developing. Bricklayer is written in Perl and will run on any platform that Perl runs on. The framework makes no assumptions about how you like to develop. You can use as much or as little of BrickLayer as you want. Think of it as a toolbox, set of raw materials, and personal construction worker as you build your app. The foundation of Bricklayer is dynamically loadable perl modules. The templating, plugins structure, error handling, database interface and even the publishing interface is built on this. In fact all Bricklayer really does is handle loading your custom modules for you and using them where appropriate. It's really simple and like many really simple concepts has a lot of power.
\section{BrickLayer's Structure}
BrickLayer ties everything together in the Bricklayer module. The file Bricklayer.pm contains the Bricklayer module. All you have to do to use Bricklayer in your Application is have a use Bricklayer; statement in your file. Once you have that you have access to all of Bricklayer's powerful plugin and templating features. Of course vanilla Bricklayer really isn't all that useful. You will have to write some custom modules first. Bricklayer has three top level divisions in it's structure.
\begin{itemize}
\item Templating
\item Plugins
\item Database Interface
\end{itemize}
\subsection*{Templating}
BrickLayer has an extremely powerful templating engine. It allows you to define custom tag identifiers and write your own handlers for the template tags themselves. In fact you will in all likelihood have to do just that since BrickLayer doesn't come stocked with a whole lot of tags to begin with. Since you can define your own tag identifiers you can even do multipass templating using the engine. Bricklayer template tags look like XML elements. They take this form: <IdentifierTagName></IdentifierTagName> and can be container or single tags. Container tags require a closing tag just like XML elements do. Single tags can close the same way that XML elements do: <IdentifierTagName /> We'll go into more detail on the templating engine, and how you can use it, in the templating Chapter.
\subsection*{Plugins}
BrickLayer has a dynamic plugin subsystem. In fact the templating engine uses some of the same concepts as the plugin subsystem uses. All Bricklayer plugins are of two general types. There are on demand loading plugins, and preloaded plugins. Out of those two general categories there are 6 plugin flavors each with their own location.
\begin{itemize}
\item access
\item action
\item content
\item data
\item error
\item publish
\item store
\end{itemize}
Each of the plugin types handles a particular type of job in the Bricklayer environment. Access plugins handle sessions and Logging in. Action plugins are general purpose workhorse plugins. Content plugins are used to filter content before publishing. Data plugins are database drivers for the database interface. Error plugins are used for error handlers and logging. Publish plugins handle actually publishing the content. You can load a plugin using the load\_plugin method. It gets called with the following syntax:
\begin{verbatim}
$pluginObj = $BK_Obj->load_plugin($Name, $Type, $ParamsRef);
$pluginObj->run($argument1, argument2, ...);
\end{verbatim}
 Plugin Development and how the plugin system works will be covered in more detail in the Plugin Chapter.
\newpage
\subsection{BrickLayer's Directory layout}
\begin{itemize}
\item[/] root directory
\begin{itemize}
\item[/] cgi (bricklayer's cgi module)
\item[/] docs (documentation)
\item[/] data (data location)
\item[/] lib (bricklayer libraries)
	\begin{itemize}
		\item[/] main
		\item[/] common
		\item[/] js (javascript libraries)
		\item[/] data (DB interface libraries)
	\end{itemize}
\item[/] plugins
	\begin{itemize}
		\item[/] access (access plugin location)
		\item[/] action (action plugin location)
		\item[/] content (content plugin location)
		\item[/] data (data plugin location)
		\item[/] error (error plugin location)
		\item[/] publish (publish plugin location)
		\item[/] store (inactive plugin location)
	\end{itemize}
\item[/] templater
	\begin{itemize}
		\item[/] handler (template tag handlers)
	\end{itemize}
\item[/] templates (template files)
\end{itemize}
\end{itemize}
\section{Using The Bricklayer Object}
\subsection*{The File Manager Object}
Bricklayer comes equipped with a file manager object to make wandering the filesystem easier without breaking the dynamic plugin loading code. You don't have to use it but if you don't you might just find your code breaking at strange times. If you use the file\_manager object though all you have to remember is to close the object when your done wandering the filesystem and you'll never get plugin or tag handler errors because they couldn't load or couldn't be found. Getting the object is as easy as calling the 
\verb|my $FileObj = $BK_Obj->new_file_obj();|
Then when your done with it make sure to call the 
\verb|$FileObj->close_obj();|
to close it. Just one more thing to remember when traversing the filesystem. If you try to call a plugin or tag handler while the \$FileObj is still open you risk having errors. Try to keep the filesystem browsing inside the block your in and close it as soon as your done. It's not too difficult really. The file\_manager object has a lot of highly useful functions Documented below:
\begin{itemize}
	\item my \$filemanger = file\_manager->new() create a file manager
	\item \$filemanger->close\_object() close a filemanager
	\item \$filemanger->current\_dir() returns current directory
	\item \$filemanger->ch\_dir(\$directory) changes to a directory
	\item \$filemanger->move\_up() moves up a directory
	\item \$filemanger->reset\_dir() resets the directory to where we started
	\item \$filemanger->mk\_dir() creates a directory in the current
	\item \$filemanger->rm\_dir() removes a directory and all files in it
	\item \$filemanger->save\_file(\$name, \$contents, \$destination) saves a file
	\item \$filemanger->test\_file(\$name) tests for a files existence in current
	\item \$filemanger->del\_file(\$name) deletes a file \$name must be a full path or the file must be in current
	\item \$filemanger->view\_dir()  view the contents of current directory
	\item \$filemanger->view\_directories() view all subdirectories of current
	\item \$filemanger->view\_file() view contents of a file
	\item \$filemanger->view\_files() view all files in current
	\item \$filemanger->view\_files\_of\_type(\$type)  view all files of a certain type eg. file extension without period
\end{itemize}
The methods are pretty self explanatory. Just make sure you close the file manager object after your done.
\subsection*{Database Interface}
Bricklayer helps all your plugins and tag handlers stay on the same page when it comes to Databases. The new\_db\_conn() method is available to all plugins through the bricklayer object for making new database connections. And the default\_db\_driver() method allows you to use a single database plugin for all your plugins automatically. The specifics of the database driver are entirely up to you however. Bricklayer does not as of yet have a standardized datbase plugin interface. This is because I haven't figured out how to write one without making certain assumptions about how and where you will store your data. Future versions of Bricklayer may add a standard inferface requirement though.
\subsection{BrickLayer Environment}
BrickLayer provides an Environment Hash for use in your application. It automatically places the Request variables from a GET or POST request in this hash. Access Plugins will place session variables in here as well. Anything placed in this hash will be available to all plugins or template tag handlers in BrickLayer. You don't have to mess with how information was sent from the browser BrickLayer does this for you behind the scenes.
\subsection*{Bricklayer Workflow}
As mentioned before the first thing you have to do to use Bricklayer is have a use Bricklayer statement in your application. Then call Bricklayer's new method to retrieve a Bricklayer object.
\begin{verbatim}
use Bricklayer;
$BK_Obj = Bricklayer->new($ConfigFileName);
\end{verbatim}
Once you have a Bricklayer object you can start using all the features. The new method to create a bricklayer object can be passed an optional argument telling it what config file to use. If no argument is passed it will default to the App.conf file in the root directory. Now lets get started using Bricklayer shall we? The Bricklayer object gives you access to all the features Bricklayer has. A typical workflow will involve
\begin{itemize}
\item[1] Retrieving any request variables.
\item[2] Running any requested actions
\item[3] Determining what template to run 
\item[4] Running a template
\item[5] Publishing your results
\end{itemize}
\par
Retrieving the request variables is as easy as looking in the Bricklayer environment hash.
\begin{verbatim}
$RequestVariable = $BK_Obj->{Env}{somevariablename};
\end{verbatim}
Any request variables will be contained in here. Whether they are POST or GET variables it doesn't matter. BrickLayer takes care of retrieving and storing them for you behind the scenes. It even stores File Uploads there. We'll cover those in more detail later on. Typically the request variables are going to tell you which template you need to run and what actions you need to do so lets get on with it shall we.
\par
Most actions will be a plugin in your action plugin directory. You can load and run them like so.
\begin{verbatim}
$pluginObj = $BK_Obj->load_plugin($Name, $Type, $ParamsRef);
$pluginObj->run($argument1, argument2, ...);
\end{verbatim}
This will load the plugin of that type and name and pass the reference to any arrays or hashes of parameters to it. You can then run the plugin with it's run method to do whatever action it is that it does. Usually the plugin type will be action but nothing says it has to be that. It may be something else if that's what you wanted.
\par
There are two ways to run a template in Bricklayer. Using a file name or just text. You can use either method but we recommend you use the file name method. It's makes things a lot easier on you. Once you know the template file to use just use the run\_templater method like this.
\begin{verbatim}
$page = $BK_Obj->run_templater($FileName, $tagid, $params);
\end{verbatim}
That will run the templating engine on your template file and return the results to you. Which you can then publish. In fact, lets do that now.
\begin{verbatim}
$BK_Obj->publish($page, $publishType,);
\end{verbatim}
The publish method publishes your page for you. It can publish the page anywhere you want it. But in this case we probably want to publish it to the browser. So we would choose the publish type that calls the plugin to send it to the browser. There is a default one of web for just that purpose. You can use the publish method to publish it to a file also or send it to a backend server. Once you pushed it out to the browser there probably isn't anything left to do. Bricklayer is really easy and really powerful.
\subsection*{BrickLayer Made Easy}
\textit{methods that tie it all together for you}\\
Bricklayer has two convenience methods that tie all of the above stuff together for you behind the scenes. This is probably the quickest way you can get started using Bricklayer. they are the execute and execute\_web methods. These methods take care of calling the action plugin, running the template, and publishing the results for you. All you have to do is call them with the proper arguments. Here is an example
\begin{verbatim}
use BrickLayer;
use strict;

eval {
my $app = BrickLayer->new;
#die "loaded Bricklayer: ";
$app->execute_web($tagid);
#$app->execute("web", $tagid); #commented out
                               #shown as an example
};
#our error trapping mechanism simple yet effective :-)
print "Content-type: text/plain\n\n
       Encountered a fatal error: $@\n" if ($@);
\end{verbatim}
execute\_web is a convenience method for automatically publishing to the web. execute is a convenience function that allows you to specify the publish plugin to use. Each of these methods takes their cues from two different request variables or values in the Bricklayer environment hash. Page and Action. The Page key in the environment hash tells them what template to run. The Action Key in the environment hash tells them what action plugin to run. The \$tagid argument in both of these gets passed to the templating engine to specify what our template tag identifiers are. They are optional and if not given will default to "BK". Now that we have seen how to use the Bricklayer object lets take a look at creating our templates, tag handlers, and plugins for use in the framework.
\chapter{BrickLayer Templating}
\section{Templating Overview}
Bricklayer templating is both extremely simple and very powerful. The Templater is called with one of two Bricklayer methods
\begin{itemize}
\item run\_sequencer(\&templatetext, \&tagid, \$paramsref);
\item run\_templater(\$templatefilename, \$tagid, \$paramsref);
\end{itemize}
run\_sequencer gets passed a scalar string of text that it will run the templating engine on. run\_templater gets passed the filename of a template that it will run the templating engine on. the other two arguments are optional. \$tagid is your template tag identifier. This value identifies your template tags. \$paramsref is a reference to an array of parameters which you can pass to your template tags. Most people won't need this but feel free to use it if you want. Now if you want to pass some parameters but don't want to pass the tagid then you'll have to pass an undefined value for tag id since it expects the arguments in that order. If you don't specify a tagid the default is BK. All template handlers must be in the templater/handler subdirectory of your BrickLayer root directory or a directory under that one.
\section{An example handler}
Now that we now how to call the templating engine we need to write a handler. The following is an example skeleton of a valid template handler. Template handlers are very simple to write. The inherit from the lib::common::template\_handler super class. They also need only one method and no package data. You can of course have more if you wish but BrickLayer doesn't need anything else to run your handler.
\newpage
\begin{verbatim}
package default;
use lib::common::template_handler;
use base qw(template_handler);

sub run {
    my $self = shift;
    my $Params = shift;
    
    return "wheeee!!!! I handled a template tag";   
}


return 1;
\end{verbatim}
The run method is required and will be passed one argument which will be an array of any parameters that bricklayer has for your handler. The Template's token object is stored in the \verb|$Self->{Token}| key. It is a has containing the following elements. It refers to the information about itself that the parser was able to determine.
\begin{verbatim}
$Token->{tagname); #The name of the template tag
$Token->{block};  #The contents of the template tag
$Token->{type); #The type of the template tag
$Token->{attributes}; #A hash of attribute name,value pairs 
                      #for the template tag
\end{verbatim}
The block or contents of the tag are unparsed by the engine. If your tag is expected to contain template text with template tags then it must call the templating engine for that text. One last detail to keep in mind when writing the template handler is binding tags to an action. When Bricklayer loads the handler for your tag it looks for an attribute called action in the tag. If that attribute exists bricklayer will automatically load and run that action plugin and store the results in the \verb|$self->{data} key| for use by the plugin. This nifty little feature will let you bind an action plugins output to a template tag.
\section{Writing Templates}
\subsection*{Template Locations}
If your going to use the run\_templater method in your app (and this is recommended) then your templates will need to be in the templates subdirectory of the Bricklayer root directory or a diectory under this one. Using the run\_templater method gives you one useful benefit. Automatic handling of subdirectories in the template directory location. Believe me if you have a lot of templates this can be a lifesaver. It does this by the simple expedient of seperating out directories with the : character. For instance a page request formatted like this: gallery:thumbnails will look for the template in the gallery subdirectory of the templates directory. Usually a Bricklayer template has an extension of .txml. This is not a hard and fast rule though so feel free to break it if you really want to. Of course if you do then the provided execute and execute\_web Bricklayer methods won't work since they expect the extension to be .txml You can nest template tags as deep as the tag handler will allow. Just remember that if you use a template tag as a container tag the handler has to parse the contents. So pay attention to what your tags will be expected to do. Template tags can occur anywhere in the template. Even though they look like XML elements they don't have to obey the rules of an xml element. They can occur inside xml or html elements. Here is an example template.
\begin{verbatim}
<html>
<head> 
	<meta content="" />
	<title>Default Index - testing</title>	
</head>
<body>

<div id="menu_main">
	<ul>
	<BKmenu>
		<li><BKmenu_item /></li>
	</BKmenu>
	</ul>
</div>

<div style="<BKstyle />" id="bklog">
	<BKbk_log />
</div>
</body>
</html>
\end{verbatim}
Note the BK identifiers on the elements. This is the default indentifier for Bricklayer templates but it is cusomizable. Also the template tags can appear inside attributes and nest inside each other. Each tag will be replaced in the document with whatever it's handler returns. Tag's can appear literally anywhere in the document. There some rules to keep in mind though. Container tags can't cross each other's boundaries or there will be an error in the sequencer. This will cause the sequencer to error for example:
\begin{verbatim}
<BKtag1>
<BKtag2>
</BKtag1> These two tags cross each others boundaries.
</BKtag2>
\end{verbatim}
Container tags must also close themselves or the results will be unpredictable.
\par
As you can see we leave you a lot of room to maneuver in the tag handlers.  Container tags will probably be used for looping and conditional processing. You can literally create your own Coldfusion like language with this engine. Feel free to use them however you feel like. They do one thing and one thing well. The rest is up to you.
\chapter{Plugin Development}
\section{Bricklayer Plugins overview}
BrickLayer Plugins come in 6 different flavors and two general types. There are on demand loading plugins, and preloaded plugins. From these two general types there are the 6 flavors:
\begin{itemize}
\item access
\item action
\item content
\item data
\item error
\item publish
\end{itemize}
\section{An Example Plugin}
We've already taken a look at what each kind of plugin handles. Now lets take a look at what you need to have a working plugin. The following page contains an example of a working skeleton for a plugin. It shows all the requried elements for a working Bricklayer Plugin. It doesn't really do anything but it will load and run in Bricklayer. A BrickLayer Plugin requires one MetaData hash, and at least the following method and library:
\begin{itemize}
\item use lib::common::plugin;
\item use base qw(plugin);
\item run();
\end{itemize}
The method and libraries are pretty self explanatory. The two use statements make your plugin a proper Bricklayer plugin by using the lib::common::plugin module as a superclass. run() gets called when Bricklayer runs the plugin automatically. This is a required method for any plugins that get loaded when Bricklayer is loaded and are run automatically. Depending on how you use Bricklayer this could be any one of the plugin types so it is advised to use this method in all your plugins as the main execution point.
\newpage
\begin{verbatim}
#---------------------------------------------
# 
# File: default.pm
# Version: 0.1
# Author: Jeremy Wall
# Definition: 
#
#---------------------------------------------
package default;
use lib::common::plugin;
use base qw(plugin);

my %MetaData = (Name => "default",
		Type => "Action",
		Author => "Author",
		Version => "0.1",
		URI => "http://someurl/",
		);

sub run {
	my $self = shift;	
	return "Whee!!!!! my plugin loaded";
}
return 1;
\end{verbatim}
\newpage
\section{The Plugin Types}
\textit{and where they are used}
\\
\subsection{Access Plugins}
Access plugins are preloaded by bricklayer. Access plugins store session information. They tell you if your logged in and what role or level you are logged in as. Access plugins have several responsibilities. The first and most important is their response to the check\_session method. Bricklayer provides the check\_session method as a means of checking the most common session values. Access plugins are expected to return the following values when the following methods are called on them like so:
\begin{verbatim}
$session{login} = $access_plugin->login; # boolean value
if ($session{login} == 1) {
	$session{name} = $access_plugin->name;
	$session{sessionid} = $access_plugin->session_id;
	$session{level} = $access_plugin->level;
	$session{role} = $access_plugin->role;
}	
\end{verbatim}
login indicates whether the current session is logged in. If it is then name, sessionid, level, and role are retrieved using their respectively named methods. The access plugin is also responsible for storing any additional session variables in the Bricklayer environment hash. There is no expected method to retrieve these. Any plugins and template tag handlers you write can use the check\_session method to see if the current session has the right to run them and respond accordingly. The check\_session method will return a session hash like the one demonstrated above. The access plugin is also responsible for storing the session information in whatever fashion you want them to. This can be in a file, database, or cookie if you prefer.
\subsection{Action Plugins}
Action Plugins are stored in the plugins/action directory. They are general purpose plugins. They will get called automatically by the execute and execute web convenience methods if you use them, or can be called at will by template tag handlers, or your application, or other plugins. Typically these plugins contain your application logic. They do the things your app is supposed to do. They may use the publish method to write to the disk. They may store things in the database. They may retrieve things from the database. In short use them for anything your feel like using them for. I will often bind an action plugin to a template tag using an attribute. These are the general purpose plugin that enables you to seperate your presentation and data from your application logic.
\subsection{Content Plugins}
Content Plugins are preloaded by Bricklayer. They get called by Bricklayer automatically during the publish method. These plugins are expected to act as filters for whatever gets published. Use them sparingly. Running too many of these could slow your app down significantly. Often these plugins will need to take a look at the environment and publish type to decide if they should run at all. The content to publish will be the first argument passed to these. Don't forget that the Bricklayer Object was passed when the plugin loaded so if you kept it then you will have access to the environment and plugins through that.
\subsection{Data Plugins}
Data plugins get loaded by the default\_db\_driver() method usually. This method loads the driver specified in the App.conf file with a line like so: dbHandler=default This driver will be used when new\_db\_conn() gets called. This can help you make sure all your plugins use the same db driver automatically. Feel free to not use it though. In the future this class of plugins will be imbued with a great deal more functionality.
\subsection{Error Plugins}
Error plugins are preloaded and get called by Bricklayer when it encounters a fatal error or wishes to log something. Error plugins will get passed a message and message type. The two predefined types are fatal and log. Fatal errors mean Bricklayer can no longer continue and must die. The error handler plugins don't have to handle this if they don't want to since Bricklayer will do it automatically. log messages mean Bricklayer just want's to preserve this for posterity. Error handlers may want to write it to a file or or store it elsewhere. Bricklayer also provides a convenient repository for log messages in the \$BK\_Obj->{Log} location. You may wish your handler to store them there. Heck you might even want a handler to provide an XML feed of fatal or log messages for you to keep track of. Why Not?
\subsection{Publish Plugins}
Publish Plugins are called by the publish method. The publish type argument passed to the publish method should correspond to a publish plugin's name. The publish plugin is then responsible for publishing the page it was passed. If the plugin can't be found then the publish method will fall back to the default web publish type.
\chapter{An Example Application}
Now that we have seen how everything works lets take a look at developing a small example application. We'll develop the oft requested oft done image gallery. Developing with bricklayer usually follows a typical workflow.
\section{Development Workflow}
When I develop in bricklayer I like to follow this pattern:
\begin{itemize}
\item[Step 1] Create our Template Structure
\item[Step 2] Write our plugins
\item[Step 3] Write our tag handlers
\item[Step 4] Rinse Repeat
\end{itemize}
\subsection{Creating our template}
The first step is to create our template. What do we want our gallery page to look like? We can fill it in with template tags we have already created or with tags we think we will need. Since this is our first bricklayer application we don't have any already created tags so we will have to create our own. The good news is once we have made them we can reuse them other places. Here is one way we might want to make our gallery page.
\newpage
\begin{verbatim}
<html>
<head>
  <title>Gallery</title>
  <BKmeta />
</head>
<body>
  
  <div>
    <BKimage_list action="list_images">
     <BKimage_row number="4">
     <img src="<BKimage attr="location" />" alt="<BKimage atr="alt" />" />
     </BKimage_row>
    </BKimage_list>
  </div>
</body>
</html>
\end{verbatim}
Granted this doesn't look like much but it will get across the basic concepts allright. Now that we have our template we need to identify the tasks our gallery will need to do and write an action plugin to do them.
\subsection{Writing our plugins}
This particular gallery will work by scanning a directory for images and displaying them in our gallery page. As such the task it will have to accomplish is scanning for the images and recording their urls for displaying them in the page. That's pretty much all it has to do so we should only need one plugin for it. Let's call our action plugin list\_images. It will go in the plugins/action directory. and just like our example plugin it requires a certain set of methods and class variables. Besides those it will have to return a list of images in a form our tag handler will be able to handle. It's probably best to just have the run method return this for us so we stay with the standard. Lets get started shall we?
\newpage
\begin{verbatim}
#---------------------------------------------
# 
# File: list_images.pm
# Version: 0.1
# Author: Jeremy Wall
# Definition: 
#
#---------------------------------------------
package list_images;

my %MetaData = (Name => "list_images",
		Type => "Action",
		Author => "Author",
		Version => "0.1",
		URI => "http://someurl/",
		);
	
sub get_type {
	return "$MetaData{Type}";
}

sub get_meta_data {
	return MetaData;
}

# Initialization

sub load {
	my $proto = shift;
	my $class = ref($proto) || $proto;
	my $App = shift; 
		
	my $PluginObj = {App => $App};	
	return bless($PluginObj, $class);	
}
	

sub run {
	my $self = shift;
	my @imagelist; # this will be our array of images
	my $bk = $self->{App}; # our bricklayer object 
	                       # for any bricklayer utilities 
	                       # we might need
	
	#code to return a list of images in the gallery directory
	
	
}
return 1;
\end{verbatim}
Now we have our plugin's skeleton all done. All we have to do is write our code to retrieve the array of images from the directory. We can use bricklayers filemanager object to do this so we avoid any unpleasantness with the dynamic plugin loading structure. What we need is to take our directory request which we will probably want to get from a request variable found in our environment hash. Good thing we stored the bricklayer app object cause we are going to need it.
\begin{verbatim}
#---------------------------------------------
# 
# File: list_images.pm
# Version: 0.1
# Author: Jeremy Wall
# Definition: 
#
#---------------------------------------------
package list_images;
use lib::common::plugin;
use base qw(plugin);

my %MetaData = (Name => "list_images",
		Type => "Action",
		Author => "Author",
		Version => "0.1",
		URI => "http://someurl/",
		);
sub run {
	my $self = shift;
	my @imagelist; # this will be our array of images
	my $bk = $self->{App}; # our bricklayer object 
	                       # for any bricklayer utilities 
	                       # we might need
	
	#code to return a list of images in the gallery directory
	my $directory = $bk->{Env}{dir}; # retrieve our gallery directory
	my $imagetype = $bk->{Env}{imagetype}; # retrieve the image type we want;
	my $fileobj = $bk->new_file_obj($directory);
	
	@imagelist = $fileobj->view_files_of_type($imagetype);
	$fileobj->close_object();
	return \@imagelist;
	
}
return 1;
\end{verbatim}
We recovered two items from the environment hash to tell us where the directory was and what kind of image files we were looking for. then we used the file manager object to retrieve the list for use and then returned it making sure to close the file manager object before we do. (\emph{we really can't emphasize this enough It's the one truly hard and fast rule in Bricklayer}) Now we are ready to start working on our tag handlers.
\subsection{Writing the handlers}
We have several tag handlers to write. The include, image\_list, image\_row, and image handlers are what our template needs. Each one has a particular task. The include handler will just include a file. The image\_list handler calls our action plugin and passes the result on to it's children handlers. The image\_row handler outputs our images in rows according to our number attribute. And finally the image handler outputs the specific information for each image. Lets get started on them shall we.
\begin{verbatim}
package image_list;
use lib::common::template_handler;
use base qw(template_handler);

sub run {
    my $self = shift;
    my $Params = shift;
    
    my $content = $self->{Token}->{block);
    my $result = $self->{data};
    my $parsed = $App->run_sequencer($content, "", $result)
       or $App->errors("failed running sequencer on $self->{Token}->{tagname} tag", "log");
    
    return $parsed;
}

return 1;
\end{verbatim}
As you can see there usually isn't a whole lot complicated to a tag handler. They just take data and pass it on or output. This particular handler retrieves the image array and passes it on to it's children tags that it runs the sequencer on. Then it returns the result of the sequencer. Now we need to write the next handler
\begin{verbatim}
package image_row;
use lib::common::template_handler;
use base qw(template_handler);

sub run {
    my $self = shift;
    my $image_array = shift;
    
    my $content = $self->{Token}->{block);
    my $loop =  $self->{Token}->{attributes){number};

    my $parsed;
    my $counter = 0;
    
    # loop as many times as our number attribute indicates
    foreach (@$image_array) {
       $parsed .= $self->{App}->run_sequencer($content, "", $result)
         or $self->{App}->errors("failed running sequencer on $self->{Token}->{tagname} tag", "log");
       $parsed .= "<br />" if $counter = 4;
       $counter .= 0 if $counter = 4;
       $counter++;
    }

    return $parsed;
}

return 1;
\end{verbatim}
This tag is only slightly more complicated than it's predecessor. It takes the array passed as a parameter and sends each element on to the children tags also outputting a <br> tag every 4 images. We only have one tag handler left. The image tag handler.
\begin{verbatim}
package image_row;
use lib::common::template_handler;
use base qw(template_handler);

sub run {
    my $self = shift;
    my $image = shift;
    my $App = $self->{App};
    my $url = $App->{Env}{dir};
    
    my $attribute = $self->{Token}->{attributes}{attr};
    if ($attribute == "location") {
      my $parsed = $url . "/$image";
    } elsif ($attribute == "location") {
      my $parsed =  "Image Name: $image";
    }
    
    return $parsed;
}

return 1;
\end{verbatim}
This handler takes our location passed on to it and outputs a url and an alternate text based on which attribute this particular handler has. This was our last handler so now we can start using our new gallery application.
\section{Conclusion}
Bricklayer does a pretty good job of staying out of your way and allowing you to work. Just remember the few ground rules and use what you feel like using. Bricklayer isn't a full fledged application. It's just a toolkit and framework to build your application on. I hope you find it useful to use.
\end{document}